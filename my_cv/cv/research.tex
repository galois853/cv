%-------------------------------------------------------------------------------
%	SECTION TITLE
%-------------------------------------------------------------------------------
\cvsection{主要研究方向}


%-------------------------------------------------------------------------------
%	CONTENT
%-------------------------------------------------------------------------------
\begin{cventries}

%---------------------------------------------------------
  \cventry
    {指导教师:横山敦士} % Advisor
    {京都工艺纤维大学} % Institution
    {日本,京都} % Location
    {2018.9 - 至今} % Date(s)
    {
      \begin{cvitems} % Description(s)
        \item {
基于遗传算法的神经网络的拓扑结构设计和复合材料强度预测。神经网络可以用来预测,
近似,分类等,神经网络的拓扑结构对于
神经网络的性能有着决定性的影响,遗传算法可以辅助寻找最佳的拓扑结构。该研究设计了一个三层的神经网络,隐藏层的神经元的数量以及隐藏神经元与
输入层的连接关系,以及使用的激活函数是神经网络拓扑结构的变量。然后对网络进行编码,使用遗传算法进行搜索。用其中性能最好的结构来
对复合材料的力学性能进行预测,从而用来取代经典复杂力学数学模型,减少计算复杂度,简化计算过程。}
        \item {基于遗传算法的多约束条件的复合材料的优化设计。遗传算法可以高效解决基于离散变量的优化问题,
	但是对于具有约束条件的目标优化,需要对目标函数进行重新构造,因为遗传算法是为了解决没有
	约束条件的问题提出来的。在提出的算法改进策略中,使用群体分类
	和自适应演化 
	技巧,从而不用对目标函数进行重新设计,完成受约束离散变量的优化设计。并基于改进的遗传算法实现层合板的优化设计。}
      \end{cvitems}
    }

%---------------------------------------------------------
  \cventry
    {指导教师:钟跃崎} % Advisor
    {东华大学} % Institution
    {上海} % Location
    {2015.9-2018.3} % Date(s)
    {
      \begin{cvitems} % Description(s)
        \item
			{基于凸包算法黏连曲线分离和基于椭圆傅立叶的曲线拟合。凸包算法可以用来获取人体横断面曲线的外包围圆,傅立叶拟合可以用来对曲线的噪声进行去除。
			使用三维扫描设备获取得到的人体点云数据进行横切,从而获得人体不同部位的人体横断面曲线。对于在在人体腋窝点处得到的横断面曲线是三个黏连的圆,
			它的分离有利于确定人体关键点,也就是肩点的位置,为了准确分离三个粘连的曲线,使用凸包算法寻找分离点。对于在人体腹部等其他部位获得的横断面曲线
			,可以用来确定人体关键部位尺寸,比如腰围,胸围等。为了准确获取人体尺寸,使用椭圆傅立叶对获得曲线进行去噪。
}
      \end{cvitems}
    }

%---------------------------------------------------------

%---------------------------------------------------------
\end{cventries}
